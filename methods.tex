\section{Method}
This will be about the DNA assembly.\\
\subsection{SynBio methods}
This will be very specifically talking about the plasmid.

\subsection{Algorithms}

\subsubsection{Probability-based Random sampling}

\subsubsection{Machine learning experimental design}



\begin{itemize}
    \item \textit{Gaussian process regression (GPR)}. Gaussian Process regression model, which predicts the label (TIR) and how uncertain we are about our prediction (confidence width); Since the sequences in provided data have the pattern that the core area is different from each other, and other areas are similar. So the kernel for Gaussian Process we are using is the sum of kernels, for core areas we use spectrum kernel with string as input directly, and for other areas we use one-hot encoding and dot product kernel for simplicity.
    \item \textit{Upper confidence bound (UCB)}. recommends sequences to test for next round.
\end{itemize}

Combine the two parts together, we apply \textit{Gaussian Process Upper Confidence Bound (GPUCB)} algorithm, which has been theoretically analysed by \textcite{srinivas2012information}, 

\textbf{Choice of kernel function:} \textit{Spectrum kernel.}
